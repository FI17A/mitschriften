\documentclass{scrartcl}

\usepackage[german]{babel}
\usepackage[utf8]{inputenc}
\usepackage{hyperref}
\hypersetup{
	colorlinks=true,
	linkcolor=blue,
	citecolor=blue,
}


\title{Öffentliche Netze}
\subtitle{Unterrichtsmitschriften}
\date{Schuljahr 2019 / 2020}

\begin{document}

\maketitle
\tableofcontents

\section{Strukturierte Verkabelung}
% todo: write

\section{Breitbandinternet}

Großer Frequenzbereich mit vielen Kanälen und hohe Übertragungsrate. Alles über 64 kbit/s ist Breitband.

\subsection{Sonderformen}
\subsection{Glasfaseranschluss}

\section{Transmission Control Protocol}
\subsection{Sliding-Window-Prinzip}

\href{https://en.wikipedia.org/wiki/Sliding_window_protocol}{Wikipedia}

\subsection{Überlastkontrolle}

TCP erkennt Überlastung durch hohe Paketverlustraten, hat sonst aber keine anderen Mechanismen, um Netzüberladung zu erkennen.
Das Problem ist: die Datagramme werden nur beim Sender und beim Empfänger als TCP Pakete interpretiert, und im restlichen Netz als verbindungslose IP Datagramme behandelt.
\\\\
Algorithmen zur Überlastkontrolle:

\begin{itemize}
	\item Slow-Start-Algorithm
	\item \href{https://en.wikipedia.org/wiki/Nagle's_algorithm}{Nagle-Algorithm}
	\item Ermitteln der RTT\footnote{Round-Trip-Time}
\end{itemize}

\noindent
Weitere informationen hierzu sind auf \href{https://en.wikipedia.org/wiki/Sliding_window_protocol}{Wikipedia} zu finden.
\\\\
Mögliches Problem dabei: \href{https://en.wikipedia.org/wiki/Silly_window_syndrome}{Silly Window Syndrome}

\end{document}
