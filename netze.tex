\documentclass{scrartcl}

\usepackage{hyperref}
\usepackage[german]{babel}
\usepackage[utf8]{inputenc}
\usepackage{hyperref}
\hypersetup{
	colorlinks=true,
	linkcolor=blue,
	citecolor=blue,
}
\usepackage[official]{eurosym}
\usepackage{enumerate}


\title{Öffentliche Netze}
\subtitle{Unterrichtsmitschriften}
\date{Schuljahr 2019 / 2020}

\begin{document}

\maketitle
\tableofcontents

\section{Transmission Control Protocol}
\subsection{Sliding-Window-Prinzip}

\href{https://en.wikipedia.org/wiki/Sliding_window_protocol}{Wikipedia}

\subsection{Überlastkontrolle}

TCP erkennt Überlastung durch hohe Paketverlustraten, hat sonst aber keine anderen Mechanismen, um Netzüberladung zu erkennen.
Das Problem ist: die Datagramme werden nur beim Sender und beim Empfänger als TCP Pakete interpretiert, und im restlichen Netz als verbindungslose IP Datagramme behandelt.
\\\\
Algorithmen zur Überlastkontrolle:

\begin{itemize}
	\item Slow-Start-Algorithm
	\item \href{https://en.wikipedia.org/wiki/Nagle's_algorithm}{Nagle-Algorithm}
	\item Ermitteln der RTT\footnote{Round-Trip-Time}
\end{itemize}

\noindent
Weitere informationen hierzu sind auf \href{https://en.wikipedia.org/wiki/Sliding_window_protocol}{Wikipedia} zu finden.
\\\\
Mögliches Problem dabei: \href{https://en.wikipedia.org/wiki/Silly_window_syndrome}{Silly Window Syndrome}

% ==============================================================================

\section{Strukturierte Verkabelung}
% todo: write

\section{Breitbandinternet}

Großer Frequenzbereich mit vielen Kanälen und hohe Übertragungsrate. Alles über 64 kbit/s ist Breitband.

\subsection{Sonderformen}
\subsection{Glasfaseranschluss}

\subsection{Kein plan, ist aber nicht mehr Glasfaser}

\begin{description}
	\item [LEX] Local Exchange -- Teilnehmervermittlungsknoten\\
		Teilnehmer und Leitungen werden an Vermittlungsstellen angeschlossen.
	\item [TEX] Transit Exchange -- Durchgangsnetzknoten\\
		Dient für Übertragung der Daten im Netz.
	\item [UNI] User Network Interface\\
		Anschluss von Teilnehmergeräten an LEX.
		\begin{itemize}
			\item Public UNI \item Private UNI
		\end{itemize}
	\item [NNI] Network Node Interface\\
		Schnittstelle zwischen Netzknoten.
\end{description}

Die Elemente verzweigen $X$, leiten $X$ um, fassen $X$ zusammen, und passen $X$ an\\mit $X$ = Datenströme
\\

\noindent
Vorfeldeinrichtungen:
\begin{description}
	\item [Multiplexer] Fassen Datenströme \underline{ohne} Signalisierung zusammen
	\item [Konzentratoren] Fassen Datenströme \underline{mit} Signalisierung zusammen
	\item [Cross-Connect (CC)] Lenken und verteilen zusammengefasste... \underline{hab ich nicht mitgekriegt}
	\item [Interworking Unit (WU)] Anpassung für Verbindung bestehender Netzwerke
\end{description}

\noindent
$\{Multiplexer\} \cup \{Konzentratoren\} = AAE$

\subsection {Teilnehmeranchluss bei Breitbandnetzen}

\begin{itemize}
	\item Alle Elemente sowie Referenzpunkte sind für breitbandige Kommunikation ausgelegt
	\item Netzabschluss durch B-NT1 (Public-UNI)
	\item ...And then the slide was gone.
\end{itemize}

\end{document}
