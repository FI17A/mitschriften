\documentclass{scrartcl}

\usepackage{hyperref}
\usepackage[german]{babel}
\usepackage[utf8]{inputenc}
\usepackage{hyperref}
\hypersetup{
	colorlinks=true,
	linkcolor=blue,
	citecolor=blue,
}
\usepackage[official]{eurosym}
\usepackage{enumerate}


\title{Wirtschaft}

\begin{document}

\maketitle

\tableofcontents

\section{Betriebswirtschaft}
\subsection{Wahrenbeschaffung}
\subsubsection{Bestellmengenplanung}

Faktoren für Höhe der Bestellmenge

\subsubsection{Bezugsquellen}

\begin{itemize}
	\item
		Konditionen
	\item
		Preisangaben
	\item
		Lagerung
	\item
		Verbrauch
	\item
		Auftragslage
	\item
		Kosten
\end{itemize}

Terminplanung Bestellzeitpunkt

\begin{itemize}
	\item
		Bestellpunkverfahren
	\item
		Bestellrythmusverfahren
	\item
		Fertigungssunchrone Beschaffung (Just in Time)
	\item
		Einzelbeschaffung
\end{itemize}

\subsection{Angebotsvergleich}

Unterteilung in:
\begin{description}
	\item [Direkt] vom Hersteller
	\item [Indirekt] über Handel / Vermittler
	\item [Zentral] eine Einkaufsstelle bzw. -Abteilung
	\item [Dezentral] jede Stelle selbst
\end{description}

\noindent Bezugskanäle:
\begin{itemize}
	\item Hersteller, Großhandel, Einkaufsverbündete der Hersteller
	\item B2B-Marktplätze (Internetportale für Gewerbetreibende)
	\item B2C-Marktplätze (Internetportale für Endverbraucher)
\end{itemize}

\section{Wirtschaftskunde}
\subsection{Besteuerung des Einkommens}

Steuerpflichtige geben zum Ende des Jahres beim Finanzamt eine Steuererklärung ab.
Das Finanzamt ermittelt die höhe der Steuern und erlässt einen Steuerbescheid.
Man unterteilt Steuerklasse 1 bis 6.

\subsubsection{Steuerklassen}

\begin{enumerate}[I.]
	\item Ledig, getrennt, etc.
	\item Alleinerziehend
	\item Verheiratete (Höheres Einkommen)
	\item Verheiratete (Vergleichbare Einkommen)
	\item Verheiratete (Geringeres Einkommen)
	\item Zweitjob / Nebenjob
\end{enumerate}

\subsubsection{Höhe der Versteuerung}

\begin{description}
	\item [0 \euro] Arbeitsloser Penner \underline{Bitte ergänzen!}
\end{description}

\subsubsection{anrechenbare Sonderausgaben}

\noindent
Von der Steuer können mehrere beruflich veranlasste Ausgaben abgesetzt werden:

\begin{description}
	\item [Arbeitsmittel] Alles, womit der berufliche Tätigkeit durchgeführt wird.
	\item [Arbeitsraum] Der Raum, in dem gearbeitet wird.
	\item [Berufsausbildung] Eine weitere Ausbildung oder ein weiteres Studium.
	\item [Berufskleidung] Spezielle Kleidungsordnung oder Schutzkleidung.
	\item [Bewerbungen] Alle Kosten für Bewerbungen.
	\item [Fahrt zur Arbeit] Pendlerpauschale.
	\item [Umschuldung] Eine berufliche Umschulung, damit der Job erhalten bleibt.
	\item [Umzug] Ein beruflicher Umzug.
	\item [Unfall] Beseitigung von Schäden.
\end{description}

Auch Spenden können von der Steuer abgesetzt werden.

Die Lohnsteuer ist eine besondere Form der Einkommenssteuer; der AG zieht diese direkt vom Lohn ab.

\end{document}
